\documentclass[a4paper]{article}


% import packages
\usepackage{graphicx}  % for \includegraphics with jpg,png,...
\usepackage[english,ngerman]{isodate, babel}  % ngerman: Neue Rechtschreibung
\usepackage{url}
\usepackage[dvipsnames]{xcolor}  % for \color
\usepackage{listings}  % source code support
\usepackage[T1]{fontenc}
\usepackage[utf8]{inputenc}
\usepackage{textcomp}  % support for symbols like €
\usepackage{lmodern}  % improve font
\usepackage{float}  % allow H option in \begin{subfigure}[H]
\usepackage{caption}
\usepackage{hyperref}
\usepackage{subcaption}
\usepackage[%
backend=biber,
sortlocale=de_DE,
natbib=true,
style=numeric,
autocite=inline
]{biblatex}
\addbibresource{references.bib}


% configure packages
\setlength{\columnsep}{8mm}

% redefine hyperref names
\addto\extrasngerman{
	\renewcommand{\sectionautorefname}{Kapitel}
	\renewcommand{\subsectionautorefname}{Abschnitt}
	\renewcommand{\subsubsectionautorefname}{Unterabschnitt}
}

\definecolor{codebg}{HTML}{eff0f1}
\lstset{
    backgroundcolor=\color{codebg},
    breaklines=true,
    tabsize=2,
    basicstyle=\small\ttfamily\bfseries,
    postbreak=\kern-5ex\mbox{\textcolor{gray}{$\hookrightarrow$}\space},
    numbers=left,
    numberstyle=\color{gray},
    showspaces=false,
    showstringspaces=false,
    tabsize=1,
    frame=single,
    rulecolor=\color{white},
    xleftmargin=3pt,
    xrightmargin=3pt,
}
% from https://github.com/cansik/kotlin-latex-listing
\lstdefinelanguage{Kotlin}{
    comment=[l]{//},
    commentstyle={\color{gray}\ttfamily},
    emph={delegate, filter, first, firstOrNull, forEach, lazy, map, mapNotNull, println, return@},
    emphstyle={\color{OrangeRed}},
    identifierstyle=\color{black},
    keywords={abstract, actual, as, as?, break, by, class, companion, continue, data, do, dynamic, else, enum, expect, false, final, for, fun, get, if, import, in, interface, internal, is, null, object, override, package, private, public, return, set, super, suspend, this, throw, true, try, typealias, val, var, vararg, when, where, while},
    keywordstyle={\color{NavyBlue}\bfseries},
    morecomment=[s]{/*}{*/},
    morestring=[b]",
    morestring=[s]{"""*}{*"""},
    ndkeywords={@Deprecated, @JvmField, @JvmName, @JvmOverloads, @JvmStatic, @JvmSynthetic, Array, Byte, Double, Float, Int, Integer, Iterable, Long, Runnable, Short, String},
    ndkeywordstyle={\color{BurntOrange}\bfseries},
    sensitive=true,
    stringstyle={\color{ForestGreen}\ttfamily},
}

\setlength{\parindent}{0cm}  % remove paragraph indentation
\setlength{\intextsep}{8pt}  % set space between two float objects

\raggedbottom  % move spacing produced by figures to the bottom of the page


\begin{document}

\title{\textbf{Dokumentation\\der Android-Anwendung\\"`stock-simulator"'}}
\author{
    Jan Müller\\
    Jonas Thelemann\\
    Juri Lozowoj\\
    Lucas Held
}
\maketitle

\begin{figure}[H]
    \centering
    \includegraphics[width=\textwidth,keepaspectratio]{./images/stock-simulator-social_sub.png}
\end{figure}

\pagebreak
\tableofcontents
\pagebreak

\begin{lstlisting}[caption={example kotlin code}, captionpos=b, label={lst:example}, language=Kotlin]
private suspend fun StockbrotQuote.executeSellOrder(quote: Quote) {
    val depotQuote = accountRepository.depotQuoteBySymbol(id) ?: return
    if (quote.latestPrice >= minimumSellPrice) {
        val amount = depotQuote.amount
        Timber.i("Bot is selling $amount for ${quote.latestPrice}")
        accountRepository.sell(quote, amount)
    }
}
\end{lstlisting}

\begin{figure}[H]
    \centering
    \includegraphics[height=10cm,keepaspectratio]{./images/quote_buy_confirmation_dialog.png}
    \caption{example image}
    \label{fig:example}
\end{figure}

\pagebreak


\section{Einleitung}
\label{sec:introduction}
% Gute Einleitung ins Thema (was war die Aufgabe, etc.)
Im Rahmen der Blockveranstaltung \textit{Code-Camp Context-Awareness 1} des Fachgebiets \textit{Kommunikationstechnik} im Fachbereich \textit{16 - Elektrotechnik/Informatik} der Universität Kassel entwickelten mehrere Teams von drei bis vier Studenten innerhalb einer Woche selbstständig ein Börsensimulatorspiel für Android-Smartphones. Dabei stellt die Verwendung von echten, aktuellen Anlagekursen eine wichtige Anforderung an das Spiel dar. Mithilfe von zur Verfügung stehendem Spielgeld soll der Benutzer die Möglichkeit haben, mit Anlagen, wie Aktien oder Kryptowährungen, zu handeln. Um den Handel mit verschiedenen Anlagen zu ermöglichen, wird eine Suchfunktion gefordert, welche die verfügbaren Anlagen filtert und anzeigt. Um das eigene Budget immer im Blick zu behalten, soll eine Depotübersicht sowie eine Anzeige für den aktuellen Kontostand und dessen Verlauf erstellt werden. Die getätigten Käufe und Verkäufe sollen in einer Historie bzw. einem Transaktionsverlauf dargestellt werden. Graphen sollen die Anlagenkurse und den Kontostandsverlauf visualisieren. Wie auf gängigen Tradingplatformen üblich, soll der Simulator auch mit jedem Kauf- oder Verkauf Transaktionskosten erheben. Das Trading soll optional durch einen Bot automatisiert werden können. Entsprechende Zielwerte für den automatisierten Handel sollen für jede Anlage individuell anpassbar sein. Damit der Benutzer die Möglichkeit hat, das Spiel neu zu beginnen, soll die Anwendung eine Option bieten, den Spielstand zurückzusetzen. Zusätzlich zu den bisher genannten Hauptfeatures, wird mindestens ein Zusatzfeature gefordert, welches eine nützliche Erweiterung für die Anwendung darstellt. Ziel des Spiels ist es, durch den erfolgreichen Handel mit Anlagemöglichkeiten das Startkapital zu vermehren.


\section{Technische Details}
\label{sec:technologies}
TODO


\subsection{Architektur}
\label{subsec:technologies:architecture}
% App Architektur (Welches Pattern habt ihr benutzt? MVVM? Livedata?) Was ist das und warum ist das cool?
% Schwierigkeiten und wie ihr sie gelöst habt
Die App-Architektur folgt offiziellen Empfehlungen\autocite{google_recommendations} und basiert auf einer \textit{Activity} sowie dem \textit{Model-View-ViewModel-Pattern} (MVVM).
Bei diesem Entwurfsmuster werden Präsentation der Benutzeroberfläche und Logik voneinander getrennt.

Benutzeroberflächen verwenden \textit{Databinding} um von \textit{ViewModels} bereitgestellte Daten anzuzeigen.
Der dafür notwendige Quelltext wird automatisch generiert, wodurch Fehleranfälligkeit reduziert und Entwicklung beschleunigt werden.
Die Verwendung von \textit{LiveData} ermöglicht zudem eine automatische Aktualisierung von Benutzeroberflächen bei Veränderung des Datenbestandes.

\textit{ViewModels} sind für das Verhalten von Benutzeroberflächen verantwortlich und beziehen ihre Daten wiederum aus \textit{Models}, welche in Form von \textit{Repositories} implementiert sind.

\textit{Repositories} dienen als \textit{Single Point of Truth}, wodurch fehlerhafte oder inkonsistente Datenbestände vermieden werden.
Sie sind an die interne Datenbank (siehe \autoref{subsubsec:technologies:bibs:room}) sowie verschiedene Netzwerkquellen (siehe \autoref{subsec:technologies:apis}) angebunden.

%unfertig, sehr raw

\subsection{APIs}
\label{subsec:technologies:apis}
Um eine Simulation mit echten Kursdaten zu ermöglichen wurden zwei Schnittstellen angeschlossen.
Sie verwenden unterschiedliche Datenstrukturen, weshalb separate Modelle implementiert wurden.
Die API-spezifischen Antworten der Schnittstellen werden jedoch zu einem einheitlichen Datenmodell transformiert, so dass das Domänenmodell der App simpel und leichtgewichtig bleibt.


\subsubsection{CoinGecko}
\label{subsubsec:technologies:apis:coingecko}
% https://www.coingecko.com/
\textit{CoinGecko}\footnote{Siehe \url{https://www.coingecko.com/}} stellt Informationen über Kryptowährungen bereit.
Dazu zählen Symbol, Name und Wechselkurse.
Bei letzteren wird nur auf den USD Wechselkurs zurückgegriffen.
Darüber hinaus sind Marktdiagrammdaten verfügbar, die für Kursgraphen verwendet werden.

Die Verwendung der \textit{CoinGecko} API erfordert eine Zuschreibung, welche im Settings-Screen vorhanden ist (siehe \autoref{subsec:functionality:settings}).

\subsubsection{IEX Cloud}
\label{subsubsec:technologies:apis:iex}
% https://iexcloud.io/
\textit{IEX Cloud}\footnote{Siehe \url{https://iexcloud.io/}} stellt allgemeine Finanzdaten zur Verfügung.
Nicht alle Daten sind kostenlos zugänglich, weshalb nur Aktieninformationen über diese Schnittstelle bezogen werden.
Analog zu \textit{CoinGecko} liefert \textit{IEX Cloud} Symbole, Namen, aktuelle Preise und Marktdiagrammdaten zu Aktien.
Darüber hinaus besitzt \textit{IEX Cloud} Informationen über kurzfristige Änderungen von Aktienkursen und bietet Sammlungen von Neuigkeiten zu einzelnen Aktien.


\subsection{Bibliotheken}
\label{subsec:technologies:bibs}
Im Laufe der Entwicklung wurden verschiedene Bibliotheken von Drittherstellern verwendet, um den notwendigen Programmierungsaufwand zu reduzieren.
Im Folgenden werden diese näher vorgestellt und ihre Vorzüge sowie Eigenschaften erläutert.


\subsubsection{Android Jetpack}
\label{subsubsec:technologies:bibs:jetpack}
Android Jetpack ist eine Sammlung von Bibliotheken, Programmen und Anleitungen. Ziel ist es, das Entwickeln von qualitativ hochwertigen Applikationen zu vereinfachen. Dies wird erreicht, indem dem Entwickler Elemente vorgegeben werden, wodurch weniger Boilerplate-Code geschreiben werden muss und komplizierte Prozesse stark vereinfacht werden \cite{android_jetpack}. Zu erkennen ist die Verwendung einer Jetpack Library an dem Namespace \textit{androidx.*}. Die Libraries sind unabhängig von der Platform API verwendbar. Dadurch wird Abwärtskompatibilität gewährleistet und der Entwickler bekommt plattformunabhängige und häufigere Updates der Libraries. Die Android Jetpack Komponenten lassen sich in vier Anwendungsgebiete, \textit{Fundation}, \textit{Architecture}, \textit{Behavior} und \textit{UI} einteilen. \textit{Fundation} Komponenten stellen beispielweise die Abwärtskompatibilität sicher, beinhalten ein Testframework und Ermöglichen das Arbeiten mit der Programmiersprache Kotlin. \textit{Architecture} Komponenten unterstützen den Programmierer, robusten, testbaren und wartbaren Quellcode zu schreiben. Beispielkomonenten für diesen Abwendungsbereich sind \textit{LiveData}, \textit{Data Binding}, \textit{Room} und der \textit{WorkManager}. \textit{Behavior} Komponenten ermöglichen den einfacheren Umgang mit Android Services, wie Benachrichtigungen, Kamera und Berechtigungen. \textit{UI} Komponenten erleichtern die Entwicklung der Benutzeroberfläche, beispielsweise durch \textit{Fragments}.


\subsubsection{Moshi}
\label{subsubsec:technologies:bibs:moshi}
% https://github.com/square/moshi
TODO


\subsubsection{Room}
\label{subsubsec:technologies:bibs:room}
% https://developer.android.com/jetpack/androidx/releases/room
Die Room-Persistenzbibliothek stellt eine Abstraktionsschicht über SQLite \footnote{Siehe \url{https://de.wikipedia.org/wiki/SQLite}} dar. Ihr Ziel ist es, Datenbankanfragen zu vereinfachen und "die volle Leistung von SQLite zu nutzen" \cite{android_room_architecture}. Die Bibliothek hilft dabei, einen Cache der Anwendungsdaten auf dem Gerät zu erstellen. Dieser fungiert als Single Source of Truth und stellt sicher, dass eine konsistente Kopie der Anwendungsinformationen zur Verfügung steht. Durch die lokale Speicherung der Daten kann der Nutzer, auch offline auf diese Inhalte zugreifen. Alle Inhaltsänderungen, die vom Benutzer vorgenommen werden, werden mit dem Server synchronisiert, sobald das Gerät wieder online ist.\newline
Room besteht(im Wesentlichen aus drei Komponenten: \textit{Database}, \textit{Entity} und \textit{Dao}. \textit{Database} enthält den Datenbankinhaber und fungiert als Hauptzugriffspunkt für die Verbindung zu den persistenten Daten der Anwendung. \textit{Entity} entspricht einer Datenbanktabelle. \textit{Dao} beinhaltet die Methoden für den Zugriff auf die Datenbank. Die Anwendung nutzt die Room-Datenbank um an die Data-Access-Objects (DAOs) zu gelangen. Anschließend werden die einzelnen DAOs verwendet, um die Entitäten aus der Datenbank abzurufen und Änderungen in die Datenbank zu schreiben. Die Entitäten werden dazu verwendet, um die entsprechenden Werte der Datenbanktabellen abzurufen und zu schreiben \cite{android_room_data_storage}. Das Beziehung zwischen drei Komponenten wird in Abbildung \ref{fig:room} dargestellt.

\begin{figure}[H]
	\centering
	\includegraphics[height=7cm,keepaspectratio]{./images/Room.png}
	\caption{Room Architektur}
	\label{fig:room}
\end{figure}

\subsubsection{Retrofit}
\label{subsubsec:technologies:bibs:retrofit}
% https://github.com/square/retrofit
TODO


\section{Funktionalität}
\label{sec:functionality}
% Welche Features habt ihr gebaut?
% Welche Screens habt ihr gebaut?
% Warum sehen die Screens so aus? (warum ist button X an Position Y) / was habt ihr euch dabei gedacht?
Im Folgenden wird die Funktionalität der App, nach Screens aufgeteilt, vorgestellt.


\subsection{Navigation}
\label{subsec:functionality:navigation}
Über die \textit{Bottom Navigation}\autocite{bottom_navigation} können Nutzer zwischen den Hauptfeatures Account, Suche, Transaktionsverlauf, Stockbrot und Errungenschaften wechseln.
Diese Art der Navigation wurde gewählt, da sie modernen Designempfehlungen entspricht und bessere Kompatibilität mit Android 10 Gesten aufweist als beispielsweise ein \textit{Hamburger Menü}.

Das Einstellungsmenü ist über das Zahnrad in der Appbar von allen Screens, au"ser den Einstellungen selbst sowie den News, erreichbar.

Weitere Navigation, wie zu den News oder zur Detailansicht eines Anlageguts, ist über Bedienelemente in Form von Buttons möglich.


\subsection{Account}
\label{subsec:functionality:account}
TODO


\subsection{Suche}
\label{subsec:functionality:search}
Das Suchen von Aktien und Kryptowährungen nach Name und Symbol ist im Search-Screen möglich.
Zudem können Nutzer dort auswählen, ob sie nur Aktien oder Kryptowährungen sehen möchten.

\autoref{fig:functionality:search:full} zeigt den Aufbau dieses Screens.
Das Eingabefeld für Suchanfragen und die Dropdown-Liste zum Filtern befinden sich am oberen Bildschirmrand.
Darunter werden Suchergebnisse in einem scrollbaren Raster angezeigt.
Diese Aktualisieren sich automatisch, sobald sich ein Suchkriterium verändert.
Dadurch ist es möglich auf Nutzereingaben unmittelbar zu reagieren und Ergebnisse schnell anzuzeigen.
Während eine Suche stattfindet wird ein Fortschritts\-indikator angezeigt, um Nutzer über den aktuellen Zustand zu informieren.

Durch Klicken auf ein Suchergebnis können Nutzer zur Detailansicht dieses Anlageguts navigieren (siehe \autoref{subsec:functionality:quote}).

\begin{figure}[H]
	\centering
	\includegraphics[height=7cm,keepaspectratio]{./images/search_type.png}
	\includegraphics[height=7cm,keepaspectratio]{./images/search_loading.png}
	\includegraphics[height=7cm,keepaspectratio]{./images/search_done.png}
	\caption{Suche}
	\label{fig:functionality:search:full}
\end{figure}


\subsection{Transaktionsverlauf}
\label{subsec:functionality:history}
TODO


\subsection{Stockbrot}
\label{subsec:functionality:stockbrot}
Der Name Stockbrot leitet sich aus den Begriffen stock (deutsch: Aktie) und Bot ab. Mithilfe dieses Bots ist der Benutzer in der Lage das Traden von Aktien oder Kryptowährungen zu automatisieren. Für jede Anlage, welche sich unter der Kontrolle des Bots befindet, wird ein sogenannter Worker alle 15 Minuten ausgeführt. Von dem Worker wird geprüft, ob die zuvor in der Detailansicht (\autoref{subsec:functionality:quote}) festgelegten Schwellwerte des aktuellen Kurses erreicht wurden. Falls dies der Fall ist, wird eine entsprechende Transaktion ausgeführt. Die Worker werden im Hintergrund mithilfe des \textit{WorkManagers} gesteuert.

\autoref{fig:functionality:stockbrot:overview} zeigt den Screen, der eine Übersicht über die von dem Stockbrot verwalteten Aktien und Kryptowährungen liefert. Die Navigation zu diesem Screen erfolgt über einen Klick auf den Eintrag "`Stockbrot"' in der Bottom Navigation. Wenn zuvor keine Aktien oder Kryptowährungen zu dem Bot hinzugefügt wurden, werden in diesem Screen keine Einträge aufgelistet (\autoref{fig:functionality:stockbrot:overview:empty}). Sobald eine Aktie/Kryptowährung der Kontrolle des Bots über die Detailansicht einer Anlage (\autoref{subsec:functionality:quote}) übergeben wurde, wird diese in als Element der Liste angezeigt (\autoref{fig:functionality:stockbrot:overview:entry}). Jedes dieser Elemente besteht aus der Id, sowie den Schwellwerten, bei deren Unter- oder Überschreitung der Bot einen Kauf- oder Verkauf veranlassen soll.

\begin{figure}[H]
    \begin{subfigure}{.5\textwidth}
        \centering
        \includegraphics[height=8cm,keepaspectratio]{./images/stockbrot_list_empty.png}  
        \caption{Stockbrot Übersicht ohne Eintrag}
        \label{fig:functionality:stockbrot:overview:empty}
    \end{subfigure}
    \begin{subfigure}{.5\textwidth}
        \centering
        \includegraphics[height=8cm,keepaspectratio]{./images/stockbrot_list.png}  
        \caption{Stockbrot Übersicht mit Eintrag}
        \label{fig:functionality:stockbrot:overview:entry}
    \end{subfigure}
    \caption{Stockbrot Übersicht}
    \label{fig:functionality:stockbrot:overview}
\end{figure}


\subsection{Errungenschaften}
\label{subsec:functionality:achievements}
Um den Benutzer des Spiels vor Herausforderungen zu stellen, sind verschiedene Errungenschaften erreichbar. Eine Übersicht über diese Erfolge, sind in dem Achievements-Screen einsehbar, welcher über den Eintrag "`Achievements"' in der Bottom Navigation zu erreichen ist. Dieser Screen listet, wie in \autoref{fig:functionality:achievements:overview:closed} sichrbar, alle bereits erreichten und noch freizuschaltenden Errungenschaften auf. Die Hintergrundfarben der noch nicht freigeschalteten Errungenschaften, werden in der Liste ausgegraut. Jedes der Elemente besteht aus einem Titel und einer Beschreibung. Die Beschreibung ist für alle bereits erreichten Errungenschaften, mittels Pfeil auf der rechten Seite des Elements, aufklappbar \autoref{fig:functionality:achievements:overview:open}.

\begin{figure}[H]
    \begin{subfigure}{.5\textwidth}
        \centering
        \includegraphics[height=8cm,keepaspectratio]{./images/achievements_list.png}  
        \caption{Errungenschaften Übersicht ohne Beschreibung}
        \label{fig:functionality:achievements:overview:closed}
    \end{subfigure}
    \begin{subfigure}{.5\textwidth}
        \centering
        \includegraphics[height=8cm,keepaspectratio]{./images/achievements_list_open.png}  
        \caption{Errungenschaften Übersicht mit Beschreibung}
        \label{fig:functionality:achievements:overview:open}
    \end{subfigure}
    \caption{Errungenschaften Übersicht}
    \label{fig:functionality:achievements:overview}
\end{figure}

Sobald eine Errungenschaft freigeschaltet wird, erscheint im unteren Teil des aktuellen Screens eine Toast-Benachrichtigung \autocite{android_toasts} (siehe \autoref{fig:functionality:achievements:reached}). Diese zeigt den Titel der Errungenschaft an. Beim nächsten Aufruf der Übersicht der Errungenschaften ist dieser soeben erreichte Erfolg nicht mehr ausgegraut und die Beschreibung lässt sich einsehen. Sollten mehrere Errungenschaften direkt hintereinander freigeschaltet werden, werden auch die Benachrichtigungen nacheinander angezeigt. Intern wird sichergestellt, dass kein Erfolgt mehrfach angezeigt wird, selbst wenn die Kriterien zur Freischaltung eines Erfolgs öfter erfüllt werden.

\begin{figure}[H]
    \centering
    \includegraphics[height=10cm,keepaspectratio]{./images/achievement_reached.png}
    \caption{Errungenschaft wurde erreicht}
    \label{fig:functionality:achievements:reached}
\end{figure}


\subsection{Detailansicht}
\label{subsec:functionality:quote}
TODO

\autoref{fig:functionality:stockbrot:addremove} zeigt einen Abschnitt aus der Detailansicht, welcher das Hinzufügen einer Anlage zu dem Bot (\autoref{fig:functionality:stockbrot:add}), sowie das Entfernen dieser (\autoref{fig:functionality:stockbrot:remove}) ermöglicht. In dem Abschnitt, beginnend mit dem Label "`Auto buy/sell"', können Kauf- und Verkaufsoptionen für den Bot festgelegt werden. Ohne Spezifikation der Kaufoptionen, wird die Anlage nicht vom Bot gekauft und ohne Spezifikation der Verkaufsoption, nicht verkauft. Wenn weder Kauf- noch Verkaufsoptionen korrekt spezifiziert wurden, bleibt der Button "`ADD TO STOCKBROT"' ausgegraut. Das Kaufverhalten des Bots kann mittels Textfeld "`Buy amount"' und "`Maximum Buy Price"' festgelegt werden. Mit diesen Optionen wird die maximale Anzahl der zu kaufenden Aktien oder Kryptowährungen, sowie der Schwellwert des Preises festgelegt, unter dem ein Kauf stattfinden soll. Mittels der Textfelder "`Minimum Sell Price"' kann festgelegt werden, welche Grenze der Preis der Anlage überschreiten muss, damit der Bot einen Verkauf durchführt. Wird eine Aktie oder Kryptowährung bereits von dem Bot kontrolliert, kann diese mittels Klick auf den Button "`REMOVE QUOTE FROM STOCKBROT"' wieder der Kontrolle des Bots entzogen werden.

\begin{figure}[H]
    \begin{subfigure}{.5\textwidth}
        \centering
        \includegraphics[height=8cm,keepaspectratio]{./images/stockbrot_add_before.png}
        \caption{Anlage zum Stockbrot hinzufügen}
        \label{fig:functionality:stockbrot:add}
    \end{subfigure}
    \begin{subfigure}{.5\textwidth}
        \centering
        \includegraphics[height=8cm,keepaspectratio]{./images/stockbrot_add_after.png}
        \caption{Anlage vom Stockbrot entfernen}
        \label{fig:functionality:stockbrot:remove}
    \end{subfigure}
    \caption{Anlage zum/vom Stockbrot hinzufügen/entfernen}
    \label{fig:functionality:stockbrot:addremove}
\end{figure}


\subsection{News}
\label{subsec:functionality:news}
TODO


\subsection{Einstellungen}
\label{subsec:functionality:settings}
Der \textit{Settings-Screen} beinhaltet Optionen zum Zurücksetzen von Spieldaten, Aktualisieren der verfügbaren Anlagegüter sowie Hinzufügen und Entfernen einer Fingerabdruck-Sperre.
Wie in \autoref{fig:functionality:settings:full} zu sehen sind Buttons der zuvor beschriebenen Optionen vertikal angeordnet.
Der Button zum Hinzufügen und Entfernen einer Fingerabdruck-Sperre ist zudem deaktiviert, falls dies nicht möglich ist.
Ein Informationstext unter diesem Button wei"st Nutzer auf den Zustand hin.

Am unteren Ende des Screens befinden sich die notwendigen Zuschreibungen an die verwendeten Schnittstellen (siehe \autoref{subsec:technologies:apis}).
Durch Klicken auf einen der Links werden Nutzer zur Website der jeweiligen Schnittstelle weitergeleitet.

\begin{figure}[H]
	\centering
	\includegraphics[height=7cm,keepaspectratio]{./images/settings.png}
	\caption{Einstellungen}
	\label{fig:functionality:settings:full}
\end{figure}


\section{Teamwork}
\label{sec:teamwork}
TODO


\section{Zusammenfassung}
\label{sec:summary}
TODO


\section{Fazit}
\label{sec:conclusion}
TODO


\section{Ausblick}
\label{sec:outlook}
% was kann man noch machen, was habt ihr nicht geschafft,..
TODO


\printbibliography[title={Referenzen}]

\end{document}
